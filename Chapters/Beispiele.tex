\chapter*{Beispiele}

Jedes Kapitel beginnt mit dem \texttt{\textbackslash Chapter{}} command. Wenn du ein neues erstellst, muss das im neuen Dokument an erster Stelle. Direkt im Anschluss das \texttt{\textbackslash chapteroverlay} -- das ist mein Styling für Kapitelseiten.

\section{Strukturierung}
Man strukturiert sein Dokument mit Befehlen wie \texttt{\textbackslash section}, \texttt{\textbackslash subsection} und \texttt{\textbackslash subsubsection}. LaTeX kümmert sich automatisch um die Nummerierung und das Inhaltsverzeichnis.

% ---------------------------
% ---------- TEXT -----------
% ---------------------------
\section{Text formatieren}
In \LaTeX schreibst du einfach deinen Text. Ein einfacher Zeilenumbruch im Code wird im PDF ignoriert. Für einen neuen Absatz lässt du im Code einfach eine Zeile frei.

\textbf{Wichtige Textbefehle}

Hier siehst du, wie kleine Befehle das Schriftbild verändern. 

\vspace{1em} % Ein bisschen Platz davor

\noindent
\textbf{1. Der normale Absatz:} \\
Einfach eine Leerzeile im Code lassen. 

Das hier ist ein neuer Absatz mit dem Standard-Einzug von LaTeX.

\vspace{1em}
\noindent
\textbf{2. Zeilenumbruch ohne Einzug:} \\
\texttt{\textbackslash noindent} sorgt dafür, dass die Zeile ganz links beginnt. \\
Mit \texttt{\textbackslash\textbackslash} erzwingst du diesen Umbruch.

\vspace{1em}
\noindent
\textbf{3. Das schmale Leerzeichen (Abkürzungen):} \\
Schreibe: \texttt{z.\textbackslash thinspace} für kleine Abstände\\
Ergebnis: z.\thinspace B. (Sieht professioneller aus als z.B. oder z. B.)

\vspace{1em}
\noindent
\textbf{4. Der geschützte Abstand (Kein Umbruch):} \\
Die Tilde (\textasciitilde) verhindert, dass der Name am Zeilenende getrennt wird. \\
Lorem ipsum dolor sit amet, consectetur adipiscing elit. Aliquam fermentum sapien dui quisati  Prof.~Dr.~Nase

\vspace{1em}
\noindent
\textbf{5. Der korrekte Gedankenstrich:} \\
zwei oder drei - erzeugen en-/emdashes:\\
emdash: Text -- ein Einschub -- Text\\
endash: Text --- ein Einschub --- Text

\vspace{1em}
\noindent
\textbf{6. Sonderzeichen escapen:} \\
Sonderzeichen müssen in \LaTeX escaped werden, da einige von ihnen für andere Befehle verwendet werden. \\
Hierfür wird der  \texttt{\textbackslash} verwendet.\\
\texttt{\%} beginnt Kommentare und kann mit  \texttt{\textbackslash\%} als normales Zeichen verwendet werden.



\begin{quote}
    \textbf{Hinweis:} LaTeX kümmert sich um 99\% des Layouts. Wenn du merkst, dass du ständig \texttt{\textbackslash\textbackslash} oder \texttt{\textbackslash noindent} benutzt, stimmt meistens etwas mit der Absatzstruktur nicht.
\end{quote}

\newpage
\subsection{Hervorhebungen und Farben}
Du kannst Text ganz einfach formatieren:
\begin{itemize}
    \item \textbf{Dieser Text ist fettgedruckt} (\texttt{\textbackslash textbf\{...\}}).
    \item \textit{Dieser Text ist kursiv} (\texttt{\textbackslash textit\{...\}}).
    \item \textcolor{red}{Dieser Text ist farbig} (\texttt{\textbackslash textcolor\{Farbe\}\{...\}}).
\end{itemize}

\begin{quote}
    \textbf{Für eigene Farben:} Wenn du eine spezielle Farbe öfter brauchst, kannst du sie in der \texttt{JASstyle.sty} mit \texttt{\textbackslash definecolor\{Name\}\{RGB\}\{R, G, B\}} oder \newline \texttt{\textbackslash definecolor\{Name\}\{HTML\}\{Hex-Werte\}}  einmalig festlegen und dann überall im Dokument über den Namen nutzen.
\end{quote}

% ---------------------------
% --------- TABELLE ---------
% ---------------------------
\section{Tabellen \& Aufzählungen}
\textbf{Aufzählung mit Bulletpoints:}
\begin{itemize}
    \item Erster wichtiger Punkt
    \item Zweiter wichtiger Punkt
\end{itemize}

\vspace{1em}
\noindent
\textbf{Aufzählung mit Nummern:}
\begin{enumerate}
    \item \textbf{Schritt 1:} Vorbereitung des Projekts.
    \item \textbf{Schritt 2:} Durchführung der Analyse.
\end{enumerate}


\subsection{Einfache Tabellen}
Tabellen wirken im Code oft kompliziert. Hier ist ein Minimalbeispiel, das ihr einfach kopieren und füllen könnt:

\begin{table}[h]
\centering
\begin{tabular}{ll} % 'll' steht für zwei linksbündige Spalten
\hline
\textbf{Key} & \textbf{Value} \\ \hline
Projektname & Praxisbericht 2026 \\
Status & In Arbeit \\ \hline
\end{tabular}
\end{table}

%-------------------------
\subsection{Größere Tabelle}
\begin{table}[h]
\centering
\renewcommand{\arraystretch}{1.5} % Erhöht den Zeilenabstand in der Tabelle für bessere Lesbarkeit
\begin{tabular}{p{3.5cm} p{3.5cm} p{6cm}} % Definierte Spalten größe
\hline
\textbf{Name} & \textbf{Code} & \textbf{Bescchreibung} \\ \hline
Name 1 & Code 1 & Die 1 Beschreibung dazu \\
Name 2 & Code 2 & Die 2 Beschreibung dazu  \\
Name 3 & Code 3 & Die 3 Beschreibung dazu  \\
Name 4 & Code 4 & Die 4 Beschreibung dazu  \\
\hline
\end{tabular}
\caption{Schnellreferenz für Textbefehle}
\end{table}

% ---------------------------
% --------- BILDER ----------
% ---------------------------
\section{Bilder}
\begin{figure}[H]
\centering
    \includegraphics[scale=0.7]{Images/magins_img.png}
    \caption{Beispielhaftes Bild mit caption}
\end{figure}

% ---------------------------
% ----------- CODE ----------
% ---------------------------
\section{Codeblocks}
\begin{lstlisting} [style=csh-clean, caption={Eine Caption für den Code}]
public ICommand DeleteDepartmentCommand { get; }

 private void ExecuteDeleteDepartment(Department dep)
        {
            _onDepartmentDeleted?.Invoke(dep);
            WorkspaceDropdown.SelectedDepartmentId = "";
            CurrentDepartment = null;
        }
\end{lstlisting}

\section{Graphs}
\begin{tikzpicture}[
  font=\rmfamily\footnotesize,
  every matrix/.style={ampersand replacement=\&,column sep=2cm,row sep=.6cm},
  source/.style={draw,thick,rounded corners,fill=yellow!20,inner sep=.3cm},
  process/.style={draw,thick,circle,fill=blue!20},
  sink/.style={source,fill=green!20},
  datastore/.style={draw,very thick,shape=datastore,inner sep=.3cm},
  dots/.style={gray,scale=2},
  to/.style={->,>=stealth',shorten >=1pt,semithick,font=\rmfamily\scriptsize},
  every node/.style={align=center}]

  % Position the nodes using a matrix layout
  \matrix{
    \node[source] (a) {A}; \& \& \\
     \& \& \node[source] (b) {V};\\
     \node[source] (c) {C}; \& \node[source] (d) {D};\\
     \& \& \node[source] (e) {E};\\
     \node[source] (f) {F}; \& \& \\      
  };

% Draw the arrows between the nodes and label them.
\draw[to] (a) -- node[midway, right]{calls event \& passes data} (c);    
  \draw[to] (c) -- node[midway,right] {passes event data} (f);
  \draw[to] (f) to[bend right=20] node[midway,below] {passes event data} (d);
  \draw[to] (f) to[bend right=20] node[midway,below] {passes event data} (e);
  \draw[to] (d)-- node[midway,below] {new data is saved} (b);

  % Draw the dotted surrounding lines and add the labels as separate nodes
  % This is necessary because the anchor of the fitted node is always center
  \node[draw,dotted,fit=(a) (c) (d) (f),inner sep=4ex,] (ACDF) {};
  \node[above=-3ex of ACDF] (ACDFt) {Logic execution};
  \node[draw,dotted,fit=(b) (e), inner sep=4ex] (BE) {};
  \node[above=-3ex of BE] (Bet) {Data storage};
\end{tikzpicture}
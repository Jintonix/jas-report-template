\chapter*{Beispiele}

Jedes Kapitel beginnt mit dem \texttt{\textbackslash Chapter{}} command. Wenn du ein neues erstellst, muss das im neuen Dokument an erster Stelle. Direkt im Anschluss das \texttt{\textbackslash chapteroverlay} -- das ist mein Styling für Kapitelseiten.

\section{Strukturierung}
Du strukturierst dein Dokument mit Befehlen wie \texttt{\textbackslash section}, \texttt{\textbackslash subsection} und sogar \texttt{\textbackslash subsubsection}. LaTeX kümmert sich automatisch um die Nummerierung und das Inhaltsverzeichnis.

\section{Text formatieren}
In \LaTeX schreibst du einfach deinen Text. Ein einfacher Zeilenumbruch im Code wird im PDF ignoriert. Für einen neuen Absatz lässt du im Code einfach eine Zeile frei.

\section*{Kurzübersicht: Text-Formatierung}

In der folgenden Tabelle findest du die wichtigsten Befehle für den täglichen Schreibfluss. Nutze diese, um dein Dokument professionell zu setzen:

\begin{table}[h]
\centering
\renewcommand{\arraystretch}{1.5} % Erhöht den Zeilenabstand in der Tabelle für bessere Lesbarkeit
\begin{tabular}{p{3.5cm} p{3.5cm} p{6cm}}
\hline
\textbf{Ziel} & \textbf{LaTeX-Code} & \textbf{Anmerkung} \\ \hline
Neuer Absatz & (Leerzeile) & Standardweg für eine logische Trennung. \\
Manueller Umbruch & \texttt{\textbackslash\textbackslash} & Erzwingt eine neue Zeile (sparsam nutzen!). \\
Kein Einzug & \texttt{\textbackslash noindent} & Unterdrückt das Einrücken der ersten Zeile. \\
Schmales Leerzeichen & \texttt{\textbackslash thinspace} & Perfekt für Abkürzungen wie \textit{z.\thinspace B.} \\
Geschützter Abstand & \texttt{\~} & Verhindert Umbruch (z.\thinspace B. \textit{Prof.~Dr.}). \\
Sonderzeichen & \texttt{\textbackslash\%} oder \texttt{\textbackslash\$}  & Sonderzeichen haben teilweise Funktionen zugewiesen und müssen escaped werden. \\ \hline
\end{tabular}
\caption{Schnellreferenz für Textbefehle}
\end{table}

\begin{quote}
    \textbf{Wichtiger Hinweis:} LaTeX kümmert sich um 99\% des Layouts. Wenn du merkst, dass du ständig \texttt{\textbackslash\textbackslash} oder \texttt{\textbackslash noindent} benutzt, stimmt meistens etwas mit der Absatzstruktur nicht.
\end{quote}

\newpage
\subsection{Hervorhebungen und Farben}
Du kannst Text ganz einfach formatieren:
\begin{itemize}
    \item \textbf{Dieser Text ist fettgedruckt} (\texttt{\textbackslash textbf\{...\}}).
    \item \textit{Dieser Text ist kursiv} (\texttt{\textbackslash textit\{...\}}).
    \item \textcolor{red}{Dieser Text ist farbig} (\texttt{\textbackslash textcolor\{Farbe\}\{...\}}).
\end{itemize}

\begin{quote}
    \textbf{Für eigene Farben:} Wenn du eine spezielle Farbe öfter brauchst, kannst du sie in der \texttt{JASstyle.sty} mit \texttt{\textbackslash definecolor\{Name\}\{RGB\}\{R, G, B\}} oder \newline \texttt{\textbackslash definecolor\{Name\}\{HTML\}\{Hex-Werte\}}  einmalig festlegen und dann überall im Dokument über den Namen nutzen.
\end{quote}

\section{Tabellen \& Aufzählungen}
Aufzählung mit Bulletpoints:
\begin{itemize}
    \item Einführung in textile Techniken und die Kettenwirkerei
    \item Bindungslehre und die Analyse von Stoffen
    \item Praktisches Arbeiten an Schulungsmaschinen zur Fehleranalyse und Bedienung
    \item Technische Kalkulationen und Wartungsmaßnahmen
\end{itemize}

Aufzählung mit Nummern:
\begin{enumerate}
    \item \textbf{Serviceunterstützung und Ersatzteilversorgung:} Ein weltweit vernetztes Team von Experten steht für alle maschinenbezogenen Fragen zur Verfügung. Die Dienstleistungen umfassen Remote Service zur Reduzierung von Stillstandszeiten sowie einen schnellen Ersatzteilservice. Über ein 24/7 zugängliches Kundenportal wird der zielgerichtete Support organisiert.
    \item \textbf{Wissensvermittlung (Know-how):} Durch langjährige Erfahrung im Textilmaschinenbau unterstützt das Unternehmen Kunden bei der Maschinenbedienung, Wartung und der Textilentwicklung.
    \item \textbf{Online-Lösungen:} Digitale Services (wie das Kundenportal und die digitalen Produkte) maximieren den Wert der Maschinen durch Konnektivität. Sie liefern wertvolle Erkenntnisse zur Optimierung der Maschinenleistung und zur Steigerung der Gesamtanlageneffektivität (OEE).
\end{enumerate}

\section{Bilder}
\begin{figure}[H]
\centering
    \includegraphics[scale=0.7]{Images/magins_img.png}
    \caption{Beispielhaftes Bild mit caption}
\end{figure}

\section{Codeblocks}
\begin{lstlisting} [style=csh-clean, caption={Child-Komponente: ruft CRUD Befehle auf}]
// Der Konstruktor ruft hier das DelegateCCommand auf:
public DepartmentEditorViewModel(ObservableCollection<DepartmentViewModel> allDepartments, DepartmentDropdownViewModel dropdownVM, Action<Department, string> onDepartmentEdited, Action<Department> onDepartmentDeleted)
        {
            ...
            DeleteDepartmentCommand = new DelegateCommand<Department>(ExecuteDeleteDepartment);

        }

public ICommand DeleteDepartmentCommand { get; }

 private void ExecuteDeleteDepartment(Department dep)
        {
            _onDepartmentDeleted?.Invoke(dep);
            WorkspaceDropdown.SelectedDepartmentId = "";
            CurrentDepartment = null;
        }
\end{lstlisting}

\section{Graphs}
\begin{tikzpicture}[
  font=\rmfamily\footnotesize,
  every matrix/.style={ampersand replacement=\&,column sep=2cm,row sep=.6cm},
  source/.style={draw,thick,rounded corners,fill=KM_teal!20,inner sep=.3cm},
  process/.style={draw,thick,circle,fill=blue!20},
  sink/.style={source,fill=green!20},
  datastore/.style={draw,very thick,shape=datastore,inner sep=.3cm},
  dots/.style={gray,scale=2},
  to/.style={->,>=stealth',shorten >=1pt,semithick,font=\rmfamily\scriptsize},
  every node/.style={align=center}]

  % Position the nodes using a matrix layout
  \matrix{
    \node[source] (a) {ChildViewModel}; \& \& \\
     \& \& \node[source] (b) {DataModels};\\
     \node[source] (c) {ParentViewModel}; \& \node[source] (d) {DataProvider};\\
     \& \& \node[source] (e) {DataViewModels};\\
     \node[source] (f) {MainViewModel}; \& \& \\      
  };

% Draw the arrows between the nodes and label them.
\draw[to] (a) -- node[midway, right]{calls event \& passes data} (c);    
  \draw[to] (c) -- node[midway,right] {passes event data} (f);
  \draw[to] (f) to[bend right=20] node[midway,below] {passes event data} (d);
  \draw[to] (f) to[bend right=20] node[midway,below] {passes event data} (e);
  \draw[to] (d)-- node[midway,below] {new data is saved} (b);
  % Draw the dotted surrounding lines and add the labels as separate nodes
  % This is necessary because the anchor of the fitted node is always center

  \node[draw,dotted,fit=(a) (c) (d) (f),inner sep=4ex,] (ACDF) {};
  \node[above=-3ex of ACDF] (ACDFt) {Logic execution};
  \node[draw,dotted,fit=(b) (e), inner sep=4ex] (BE) {};
  \node[above=-3ex of BE] (Bet) {Data storage};
\end{tikzpicture}